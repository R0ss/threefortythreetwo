\documentclass[12pt]{article}
%\usepackage[T5,T1]{fontenc}
\usepackage{amssymb,amsmath}
%\usepackage{txfonts}
%\usepackage{fouriernc}
%\usepackage{ccfonts}
%\usepackage[varg]{txfonts}
%\usepackage{mathptmx}
%\usepackage{pxfonts}
%\usepackage{mathpazo}
%\usepackage{mathpple}
%\usepackage[charter]{mathdesign}
%\usepackage[utopia]{mathdesign}
%\usepackage{fourier}

%% Sanserif fonts
%\usepackage{arev}

%\usepackage{alltt} %%%%%%%%%%%%%%%%%%%%%%%%%%%% Use math in verbatim mode

%\usepackage{graphicx}
%\usepackage{rotating}
\usepackage{framed}
% This lets you box text: e.g.
% \begin{framed}
% Put your text here. It'll appear in a box.
% \end{framed}

%\usepackage{fullpage} % smaller margins.
\frenchspacing % Shortens space between 2 sentences.
\setlength{\parindent}{0pt} % No par indent
\setlength{\parskip}{2ex} % 2 lines between pars.
\linespread{1.0} % Single-spaced. Change to 2.0 for double space, etc.

% This command lets you define new math operators:
\DeclareMathOperator{\stab}{stab}

% Eg for defining TeX functions:
%\newcommand{\mnom}[2]{\left(\negthickspace\binom{#1}{#2}\negthickspace\right)}% This one's used to denote ((n ; k)) := (n+k-1 ; k) = the number of ways to put k identical balls into n distinct boxes = the number of k-multisets on an n-set.

\renewcommand{\bf}{\bfseries}

%%%%%%%%%%% SYMBOLS %%%%%%%%%%%%%%%
\newcommand{\R}{{\bf R}}
\newcommand{\Q}{{\bf Q}}
\newcommand{\Z}{{\bf Z}}
\newcommand{\N}{{\bf N}}
\newcommand{\C}{{\bf C}}
\newcommand{\e}{\epsilon}
\newcommand{\f}{\phi}
\newcommand{\s}{\sigma}
\renewcommand{\r}{\rho}
%\newcommand{\G}{\Gamma}
\newcommand{\q}{\quad}
\newcommand{\nsub}{\trianglelefteq}
\newcommand{\<}{\langle}
\renewcommand{\>}{\rangle}
\newcommand{\inv}{^{-1}}

%%%%%%%%%%%%%%% DOCUMENT-SPECIFIC COMMANDS
\newcommand{\Customer}{C}
\newcommand{\Card}{K} 
\newcommand{\Order}{O}
\newcommand{\Item}{I}
\newcommand{\OrderItem}{U}
\newcommand{\Rating}{R}
\newcommand{\CID}{\text{CID}}
\newcommand{\IID}{\text{IID}}
\newcommand{\OID}{\text{OID}}
\newcommand{\email}{\text{email}}
\newcommand{\name}{\text{name}}
\newcommand{\phone}{\text{phone}}
\newcommand{\address}{\text{address}}
\newcommand{\cI}{\text{cI}}
\newcommand{\cN}{\text{cN}}
\newcommand{\expiry}{\text{expiry}}
\newcommand{\n}{\text{number}}
\newcommand{\category}{\text{category}}
\newcommand{\IIDi}{\text{IID1}}
\newcommand{\IIDii}{\text{IID2}}
\newcommand{\quantity}{\text{quantity}}
\newcommand{\price}{\text{price}}
\newcommand{\timestamp}{\text{timestamp}}
\newcommand{\ti}{\text{t1}}
\newcommand{\tii}{\text{t2}}

\newcommand{\HRule}{\rule{\linewidth}{0.5mm}}

%% This one lets you rotate a symbol:
%\newcommand{\GG}{\begin{sideways}\begin{sideways}$\G$\end{sideways}\end{sideways}}

%\nmid = does not divide

%%% Physics %%%
%\newcommand{\Favg}{\F_\text{avg}}
%\newcommand{\Fnet}{\F_\text{net}}

%\usepackage{color}
%\pagecolor{black}
%{\color{white}
\begin{document}
%\title{CSC343 A1}
%\author{Trong Truong \& Ross Gatih\\
%		\ \ \ 995 94 222 6 \ \ \ \  997 92 311 8}
%\date{Thursday 11 October 2012}
%\maketitle
%\tableofcontents

%\begin{titlepage}
	\begin{center}
	
	% Upper part of the page
	%\includegraphics[width=0.15\textwidth]{./logo}\\[1cm]    
	
	\textsc{\LARGE CSC343 A3}\\[1.0cm]
	
	%\textsc{\Large Final year project}\\[0.5cm]
		
	% Title
	%\HRule \\[0.4cm]
	{\huge \bfseries ER MODELLING AND \\ \vspace{6pt} DATABASE DESIGN}\\[1.0cm]
	%\HRule \\[1.5cm]
	
	%\vfill
	
	% Author and supervisor
	\begin{minipage}{0.4\textwidth}
	\begin{flushright} \large
	Ross Gatih\\
	997 92 311 8
	\end{flushright}
	\end{minipage}\hspace{24pt}
	\begin{minipage}{0.4\textwidth}
	\begin{flushleft} \large
	Trong Truong\\
	995 94 222 6
	\end{flushleft}
	\end{minipage}\\[0.5cm]
	
	%\vfill
	
	% Bottom of the page
	{\large Sunday 18 November 2012}
	
	\end{center}
%\end{titlepage}

\tableofcontents

\part{Assumptions}

\section{Setup and login phase}

{\bf Conference admin info.} For these and attributes of other entities and relationships, please see ER diagram in the next part.

{\bf File storage.} We'll only store filenames of papers, forms, letters in the database. The files themselves will be stored on a hard drive somewhere outside the database.

\section{Submission phase}

{\bf Unique paperID.} Every paper will have a unique paperID across all conferences. This will remove the need to have Paper as a weak entity depending on Conference.

{\bf Different submissions are different papers.} An author has the option to update an existing submission instead of submitting an identical copy, but that's application, not database-level design, so we don't have to worry about that.

{\bf Author VS coauthor.} We don't distinguish between authors and coauthors: everyone author of a paper is a coauthor.

\section{Reviewing phase}

{\bf Number of reviewers.} Each conference has a fixed number of reviewers assigned per paper.

{\bf Reviewer expertise.} Each reviewer may have multiple areas of expertise, which correspond to paper topics.

{\bf Reviewer paper preference.} A reviewer may express preference for a particular paper.

{\bf Bidding process.} Since we have no way of knowing what the process is---each conference may have their own odd way of doing things---we can only care about the result of that process, which we assume to be, for each paper, a list of points assigned to reviewers who have expressed an interest in that paper. Again, how those points are scored is a mystery: they could be picking numbers out of a hat. The reviewers with the highest scores get to review the paper.

\section{Decision and publishing phase}

{\bf One letter per paper.} The chair sends only one acceptance notification letter per paper, to the coauthor who submitted it.

\part{ER diagram}



\part{Relational schema}



\part{PostgreSQL database definition}



\end{document}%}

